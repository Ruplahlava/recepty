\documentclass[10pt,a4paper]{article}
\usepackage[utf8]{inputenc}
\usepackage[czech]{babel}
\usepackage[T1]{fontenc}
\usepackage{lmodern}
\usepackage[margin=0.5in]{geometry}
\usepackage{hyperref}
\usepackage{gensymb}
\hypersetup{
    colorlinks,
    citecolor=black,
    filecolor=black,
    linkcolor=black,
    urlcolor=black
}
\newenvironment{myitemize}
{ \begin{itemize}
    \setlength{\itemsep}{0pt}
    \setlength{\parskip}{0pt}
    \setlength{\parsep}{0pt}     }
{ \end{itemize}                  } 
\begin{document}
\title{Recepty}
\maketitle
\tableofcontents
\pagebreak
\section{Grill}

\subsection{Pikantní asijské kuře}
\begin{minipage}[t]{0,5\textwidth}
\begin{myitemize} 
  \item Čerstvý zázvor 
  \item Med
  \item Rybí omáčka - tímhle nešetřit, bacha hodně slané
  \item Chilli
  \item Česnek
  \item Limetka
  \item Pepř, sůl 
  
\end{myitemize}
\end{minipage}
\begin{minipage}[t]{0,5\textwidth}
Křidýlka, stehýnka, prsíčko nedoporučuji. Péct podlité vodou, nebo grilovat.

Nechat den odležet. Kořením rozhodně nešetřit
\end{minipage}



\subsection{Pikantní vepřové}
\begin{minipage}[t]{0,5\textwidth}
\begin{myitemize} 
  \item Krkovička, bůčíček, kotletka... 
  \item Jablečný mošt, jablka...
  \item Hnědý cukr
  \item Sójovka
  \item Cibulka na drobno
  \item Chilli
  \item Pepř hodně, sůl 
  
\end{myitemize}
\end{minipage}
\begin{minipage}[t][5cm][t]{0,5\textwidth}
Naložit den dopředu, jinak než na gril s tím nemám zkušenosti, možná takhle naložený bůček do udírny 
\end{minipage}

\subsection{Pikantní hamburger}
\begin{minipage}[t]{0,5\textwidth}
Pohlreich
\begin{myitemize} 
  \item Mleté hovězí maso
  \item zelený pepř
  \item Čerstvý tymián, nasekaný
  \item Worcestrová omáčka
  \item Francouzská hořčice
  \item Pepř čerstvě mletý, mořská sůl 
  
\end{myitemize}
\end{minipage}
\begin{minipage}[t]{0,5\textwidth}

\end{minipage}

\subsection{Sladkokyselá žebírka po thajsku}
\begin{minipage}[t]{0,5\textwidth}
\begin{myitemize} 
  \item 2kg žeber bez kůže
  \item cibulka najemno
  \item 2 stroužky česneku
  \item zázvor
  \item 125ml ananasové šťávy
  \item slunečnicový olej
  \item 2 lžíce rybí omáčky
  \item 4 lžíce protlaku
  \item 4 lžíce limetkové šťávy
  \item 2 lžíce medu
  \item 6 žlic thajské sladké chilli omáčky
  
\end{myitemize}
\end{minipage}
\begin{minipage}[t]{0,5\textwidth}
Žebra uvařit cca 30 min ve vodě, počkat až zchladnou.
Mezitím veškeré koření připravíme na pánvi. Naložit a grilujeme.
\end{minipage}

\subsection{Limetková křídla}
\begin{minipage}[t]{0,5\textwidth}
\begin{myitemize} 
  \item 2 posekané limetky bez kůry
  \item 2 stroučky česneku
  \item 2 stroužky česneku
  \item chilli
  \item paprika
  \item třtinový cukr
 
\end{myitemize}
\end{minipage}
\begin{minipage}[t]{0,5\textwidth}
\end{minipage}

\subsection{Tex-mex vepřová žebra}
\begin{minipage}[t]{0,5\textwidth}
\textbf{Na "vývar"}
\begin{myitemize} 
  \item 1 cibule, rozčtvrcená
  \item palice česneku, přepůlená
  \item 90g třtinového cukru
  \item 1 lžíce mleté papriky
  \item 1 lžíce mleté uzené papriky
  \item lžíce soli
  \item 4 snítky tymiánu
  \item 125ml jablečného octa
  \item 1 lžíce celého pepře
  \item 2kg vepřových žeber
  \item limetka k podávání
\end{myitemize}

\end{minipage}
\begin{minipage}[t]{0,5\textwidth}
\textbf{Na BBQ marinádu}
\begin{myitemize} 
  \item 1 cibule nahrubo
  \item 2 stroučky česneku
  \item 90g třtinového cukru
  \item 90g medu
  \item 1 lžíce mleté uzené papriky
  \item 1 lžíce mleté papriky
  \item 60ml worchestrové omáčky
  \item 2 lžičky dijonské hořčice
  \item lžička chilli 
  \item 400g konzerva krájených rajčat
  \item lžíce soli
\end{myitemize}

\end{minipage}
\vspace{0.5cm}

Do velkého hrnce dát suroviny na vývar a přivést k varu. Stáhnout na střední plamen a vložit žebra. Vařit 30 - 40 minut doměkka.

Mezitím udělat BBQ marinádu. Do robotu dát cibuli, česnek, rajčata a rozsekat nadrobno, ale ne na kaši. Spolu s ostatními přísadami dát do rendlíku a přívést k varu. Ztlumit plamen a vařit 20-25 minut do zhoustnutí.

Horká žebra zalít omáčkou a důkladně potřít. Nechat marinovat přes noc.
\subsection{Grilovaná kukuřice dle Toma}
\begin{minipage}[t]{0,5\textwidth}
\begin{myitemize} 
\item Kukuřice
\item Limetková šťáva a kůra
\item Chilli omáčka
\item Kokosové mléko (malé)
\item Sojovka
\item Petrželka najemno
\item Třtinovej cukr
\item Oříšky na posyp (nesolené buráky) 
\end{myitemize}
\end{minipage}
\begin{minipage}[t]{0,5\textwidth}
Vše krome oríšků smíchat. Oříšky posekat najemno. Kukuřice namočíme v omáčce a ugrilujeme. Polijeme znovu omáčkou a posypeme ořříšky. Sežrat.
\end{minipage}
\pagebreak
\section{Pečivo, Sladké}
\subsection{Rychlokvašky}
\begin{minipage}[t]{0,5\textwidth}
\begin{myitemize} 
\item  1/2 hl mouky
\item kvasnice
\item 3 lžíce cukru
\item 2 dcl oleje
\item 2 žloutky
\item sůl
\end{myitemize}
\end{minipage}
\begin{minipage}[t]{0,5\textwidth}

\end{minipage}
\subsection{Košíčky}
\begin{minipage}[t]{0,5\textwidth}
\textbf{Těsto}
\begin{myitemize} 
\item 30dkg hl. mouky
\item 8dkg cukru
\item 20 dkg tuku
\item 2 žloutky
\item citronová kůra
\item vanilka
\end{myitemize}

\end{minipage}
\begin{minipage}[t]{0,5\textwidth}
\textbf{Náplň}
\begin{myitemize} 
\item 20dkg krystalu
\item 25dkg stuženého tuku
\item 1 kondenzované slazené mléko
\item 10dkg  nasekaných ořechů
\end{myitemize}
Rozpustí se stužený tuk a nachystá zbytek.
Rozpustí se cukr v suchém hrnci až zhnědne, přidá se tuk, mlíko, povaříme a přidáme oříšky, vaří se tak dlouho až vznikne hladká kašovitá hmota. Horké plníme do upečených košíčků. Necháme vystyddnout a polijeme čokoládovou polevou.
\end{minipage}
\subsection{Marokánky}
\begin{minipage}[t]{0,5\textwidth}
\begin{myitemize} 
\item 15dkg směsi sušených švestek, meruněk, brusinek
\item 10 dkg ořechů na plátky
\item 8dkg másla
\item 1/4l mlíka
\item 14dkg cukru
vánoční dávka jsou 4 dávky
\end{myitemize}
\end{minipage}
\begin{minipage}[t]{0,5\textwidth}
Vaříme 10 minut, do teplého přidáme 5 dkg hladké mouky. Na pomaštěný plech děláme placičky.
\end{minipage}
\subsection{Rumové kuličky}
\begin{minipage}[t]{0,5\textwidth}
\begin{myitemize} 
\item 10dkg hrozinek, krájených
\item 4 lžíce rumu
\end{myitemize}
\end{minipage}
\begin{minipage}[t]{0,5\textwidth}
nechat do rána, přidat 10 dkg cukru, 10dkg ořechů a lžičku másla, kakao. Děláme kuličky
\end{minipage}
\subsection{Opilý izidór}
\begin{minipage}[t]{0,5\textwidth}
\textbf{Těsto 1}
\begin{myitemize} 
\item Do mísy dáme:
\item 15 dkg másla 
\item 15 cukru
\item 4 žlutky
\item 1 lžičku kakaa
\item Třeme do pěny,  potem přidáme:
\item 12 ořechů
\item 7 hladké mouky
\item 1/4 prdopeč
\item sníh ze 4 bílků
\end{myitemize}
Pečeme pomalu 
\end{minipage}
\begin{minipage}[t]{0,5\textwidth}
\textbf{Těsto 2}
\begin{myitemize} 
\item 20 ořechů
\item vanilkový cukr
\item dvoudecák rumu
\item Těsto I potřeme marmeládou a těstem II
\end{myitemize}
Nakonec dáme krém:
\begin{myitemize} 
\item 4-6 žloutků 
\item 10 cukru
\item 3 lžičky až lžíce kakaa
\end{myitemize}
šleháme v páře

Vyšleháme 20 - 25 dkg másla a po malých částech přidáváme směs
\end{minipage}
\subsection{Maričina buchta}
\begin{minipage}[t]{0,5\textwidth}
\begin{myitemize} 
\item 15 dkg mletého cukru
\item 1/8 másla
\item 3 vejce
\item 2 tvarohy
\item 8dkg krupice hrubé
\item 1/2 prášku do pečiva
\end{myitemize}
\end{minipage}
\begin{minipage}[t]{0,5\textwidth}
plech silně vymazat, vysypat.Posypátko + ovoce. Trouba 200.
\end{minipage}
\subsection{Ořechové řezy - Iva. B.}
\begin{minipage}[t]{0,5\textwidth}
\begin{myitemize} 
\item 21dkg hladké mouky
\item 14 másla
\item 3 žloutky
\item 7 cukru
\item3 bíky vyšlehat a smíchat s 9 ořechy a 14 cukru
\end{myitemize}
\end{minipage}
\begin{minipage}[t]{0,5\textwidth}
uděláme těsto a vyválíme na celý plech (vymazat vysypat) trochu upečeme, apk natřeme džemem, a potřeme smesí z bílků. ihned krájet
\end{minipage}
\subsection{Dvoubarevka}
\begin{minipage}[t]{0,5\textwidth}
\textbf{Těsto 1}
\begin{myitemize} 
\item 2 lžíce kakaa
\item 1 hrnek hladké mouky
\item 1 hrnek hrué mouky
\item 1 hrnek mléka
\item 1 hrnek cukru
\item půl hrnku oleje
\item prdopeč
\item 2 vejce
\end{myitemize}
\end{minipage}
\begin{minipage}[t]{0,5\textwidth}
\textbf{Těsto 2}
\begin{myitemize} 
\item 2 tvarohy
\item 1 vanilkový cukr
\item 2 vejce
\item 1 hrnek mléka
\item vanilkový pudink
\item 3/4 hrnku cukru
\item rozinky
\item rum
\end{myitemize}
vysypat vymaštěný plech, nalít půl těsta 1, nalít T2 a zbytek T1
\end{minipage}
\subsection{Štěpánkova dorta po prababičce Hájkové}
\begin{minipage}[t]{0,5\textwidth}
\begin{myitemize} 
\item 21dkg cukru
\item 7dkg ořechů
\item 10 dkg jemné krupice 
\item 5 žloutků
\item štáva z půlky citroůnu
\item sníh z 5 bílků
\item péct asi 40 min
\item 1/2 prdopeč
\end{myitemize}
\end{minipage}
\begin{minipage}[t]{0,5\textwidth}
Žloutky cukr citrón š. se vyšlehá do pěny. Smícháme opatrně se strouhanými ořechy s moukou a prdopečem a bílky. Nalijeme do vymazané formy. Střední trouba.

Vyšleháme 3 vejce, 11 dkg cukru a kakao vyšleháme ve vodní lázni do tuha. Vyhcladnout. Smícháme s 250 vyšlehaného másla. Všechny inkredience by měly mít stejnou teplotu.

Prořízneme, potřeme merunkovou marmeládou a krémem.
\end{minipage}
\subsection{Ořechové řezy}
\begin{minipage}[t]{0,5\textwidth}
\begin{myitemize} 
\item 14 hrubé mouky
\item 14 dkg cukru
\item 10 másla
\item 6 ořechů
\item 2 vejce
\item 2 žloutky
\item sníh z 2 bílků
\item prdopeč
\end{myitemize}
\end{minipage}
\begin{minipage}[t]{0,5\textwidth}
Děláme z dvou dávek. Utřeme cukr, žloutky, máslo, smíchat s mokukou a prdopečem ořechy a se sněhem.
\end{minipage}
\subsection{Ořechové vánoční rohlíčky}
\begin{minipage}[t]{0,5\textwidth}
\begin{myitemize} 
\item 14 cukru
\item 44 hladké mouky
\item 24 másla
\item 14 strouhaných ořechů
\item vanilkový cukr
\end{myitemize}
\end{minipage}
\begin{minipage}[t]{0,5\textwidth}

\end{minipage}
\subsection{Ořechy vánoční}
\begin{minipage}[t]{0,5\textwidth}
\textbf{Těsto}
\begin{myitemize} 
\item 25 mouky
\item 20 ořechů
\item 15 cuklru
\item 20 másla
\item kakao na obarvení
\end{myitemize}
\end{minipage}
\begin{minipage}[t]{0,5\textwidth}
\textbf{Náplň}
\begin{myitemize} 
\item pudink s kakovokávovou + kostku marmelády
\end{myitemize}
\end{minipage}
\subsection{Dášina perníková buchta s třešněmi}
\begin{minipage}[t]{0,5\textwidth}
\begin{myitemize} 
\item 1/4 kg hl mouky
\item 1/4 cukru
\item 1 vejce
\item 1 lžicčka sádla nebo 3 lž. oleje
\item prodpeč
\end{myitemize}
\end{minipage}
\begin{minipage}[t]{0,5\textwidth}

\end{minipage}
\subsection{Mák od Babičky Riebelové}
\begin{minipage}[t]{0,5\textwidth}
\begin{myitemize} 
\item mák
\item cukr
\item skořice
\item povidla
\item mlíko
\item vanilka
\item rum
\item tuk (hera)
\end{myitemize}
\end{minipage}
\begin{minipage}[t]{0,5\textwidth}
Namelu mák, dám ho do vody a povařím, 1:1. Až to změkne, přidám dle chuti cukr, skořici, naliju mlíko a znova povařím (cca 5 min). Pak přidám povidla a vařím dál, dle hustoty přidám mléko a kofetu (???). Na chuť rum, vanilku, trochu tuku (Hery). Vařím do hodně husta, nesmí se to připálit! Trvá to dlouho. 
\end{minipage}

\subsection{Česnekové trojúhelníčky}
\begin{minipage}[t]{0,5\textwidth}
\textbf{Těsto}
\begin{myitemize} 
\item 750 g hladké mouky
\item 500 ml mléka
\item 100 ml oleje
\item 1 lžička cukru
\item 2 lžičky soli
\item 50 g čerstvého droždí
\end{myitemize}
\textbf{Náplň}
\begin{myitemize} 
\item 150 g změklého tuku (máslo, Hera)
\item 12 - 14 větších stroužků česneku
\item sůl dle chuti
\end{myitemize}
\end{minipage}
\begin{minipage}[t]{0,5\textwidth}
Z potřebných ingrediencí vypracujeme vláčné těsto a necháme v teple kynout do zdvojnásobení objemu. Vykynuté rozdělíme na 4 díly. Každý vyválíme na placku o síle asi 3 mm vysokou tak, aby se vám vešly všechny po zavinutí do nohavice na 1 plech.

Placky potřeme vymíchanou směsí česneku, soli a tuku. Menší část necháme stranou na potření. Smotáme do závinu a přeneseme na plech vyložený pečícím papírem anebo vymaštěný tukem. Každou nohavici po přendání na plech rádýlkem (nebo co vám vyhovuje, já použila teflonovou obracečku) překrájíme do tvaru trojúhelníků. Když máme hotové všechny 4 kusy, potřeme zbytkem nápně a ještě 30 minut necháme kynout. Poté pečeme ve vyhřáté troubě na 175 \degree C zhruba 20–30 minut.

Tip: Do nápně můžete přidat sezamové semínko a ještě s ním záviny před pečením posypat. Navrch se dá použít i mák.
\end{minipage}
\subsection{Ovesné tyčiny s blumami}
\begin{minipage}[t]{0,5\textwidth}
\begin{myitemize} 
\item 450 Blum nahrubo nakrájené
\item 1\textbackslash 2 perníkového koření
\item300 třtinový cukr
\item350 másla
\item300 ovesné vločky
\item140 hlm mouky
\item50 čokolády
\item3 lžíce medu
\end{myitemize}
\end{minipage}
\begin{minipage}[t]{0,5\textwidth}
trouba na 180, blumy zasypat kořením 2 lžice cukru a soulí - macerovat

kastrólu rozpustit máslo, v míse rozmíchat vločky mouku ořechy a zbytek cukr

máslem vymazat formu půlu směsi naspod, blumy a zbytek směsi

péct 40-50 min
\end{minipage}
\subsection{Štrůdl}
\begin{minipage}[t]{0,5\textwidth}
\begin{myitemize} 
\item 1\textbackslash 2 kg mouky (hl,pol)
\item 1 prdopeč
\item 125g jogurt
\item 1 Hera - rozpuštěná
\item 1 Vejce
\end{myitemize}
\end{minipage}
\begin{minipage}[t]{0,5\textwidth}

\end{minipage}

\subsection{Tvarohové knedlíky}
\begin{minipage}[t]{0,5\textwidth}
\begin{myitemize} 
\item 250 měkkého tvarohu
\item 200 - 250 hrubé mouky
\item 100 másla
\item jedno vejce
\item lžička prdopeč
\end{myitemize}
\end{minipage}
\begin{minipage}[t]{0,5\textwidth}
Vypracovat a vyválet těsto, naplnit ovocem, zabalit a vařit 10 - 12 minut.
\end{minipage}

\subsection{Crepes Suzette}
\begin{minipage}[t]{0,5\textwidth}
\begin{myitemize} 
\item 100 hladké mouky
\item špetku soli
\item 1 lžíci hnědého krupicového cukru
\item 2 velká vejce
\item 1 lžíci slunečnicového oleje do těsta
\item 300 ml polotučného mléka
\item asi 2 lžíce světlého piva (nemusí být)
\item rostlinný olej na potírání pánve

\end{myitemize}
\end{minipage}
\begin{minipage}[t]{0,5\textwidth}
 Mouku prosijte se solí do mísy, vsypte cukr. Uprostřed udělejte důlek, vlijte do něj vejce, olej a dvě lžíce mléka. Vařečkou prošlehejte základ těsta dohladka. Pomalu přilévejte další mléko a stále míchejte, aby směs zůstala hladká. Potom vlijte – už nemusíte tak pomalu – zbylé mléko; výsledná směs by měla připomínat (dvanáctiprocentní) smetanu. Pokud chcete, můžete nakonec přilít pivo.

2. Rozehřejte pánev (nejlépe o průměru 15 cm, což odpovídá klasickému průměru palačinek). Těsto na ni nalévejte nejlépe z naběračky s hubičkou nebo ze džbánku; na jednu palačinku by měly stačit dvě a půl lžíce těstíčka. Krouživým pohybem ho stejnoměrně rozestřete po dnu pánve. Jakmile palačinka zezlátne, což trvá asi 15 vteřin, obraťte ji a opékejte po druhé straně tak půl minuty, až na ní naskáčou typické hnědé puchýřky.
\end{minipage}
\subsection{Šnekové bulky s balzamikem}
\begin{minipage}[t]{0,5\textwidth}
\begin{myitemize} 
\item 50ml balzamikového octa
\item 350ml světlého piva 
\item 400ml vlažné vody
\item 1kg hl mouky
\item 42g čerstvého droždí
\item 15g soli
\item hl mouka na pomoučení
\item dýňová semínka (není nutné)
\end{myitemize}
\end{minipage}
\begin{minipage}[t]{0,5\textwidth}
Pokuď používáte kuchyňský robot s hákem na hnětení těsta, vlijte do mísy pivo, vodu a ocet, prosetou mouku. Hníst na střední rychlost 3 minuty. Poté přidat rozdrobené droždí a hníst dalších 7 minut na vysokou rychlost. Poté přidat sůl a hníst další 3 minuty. Přikrýt utěrkou a nechat 40-60  minut kynout na teplém místě.

Vykynuté těsto vyklopit na řádně pomoučený vál. Těsto pomoučit, prohníst a přidat polovinu semínek. Těsto roztáhnout do čtverce 40 * 40 a rozkrojit na čtyři části. Tu každou ještě na půl a stočit do tvaru šneka. Vyskládat na plech a nechat ještě 10 minut kynout. 

Vrazit do předehřáté trouby na 180 \degree C, péct asi 45 minut dozlatova.
\end{minipage}

\subsection{Koláč s makovou nádivkou} 
\begin{minipage}[t]{0,5\textwidth}
\textbf{Těsto}
\begin{myitemize} 
\item 250g mouky
\item 120g cukr krupice
\item jedem balíček vanikového cukru
\item 150g másla
\item 1 žloutek
\end{myitemize}

\textbf{Náplň}
\begin{myitemize} 
\item 500ml mlíka
\item 150g krupicového cukru
\item 2 balíčky pudinku
\item 300g máku
\item 1 bílek
\item 1 mandarinkový kompot
\end{myitemize}
\end{minipage}
\begin{minipage}[t]{0,5\textwidth}
Ze všech ingeriencí si připravíme drobenku. Mouku prosijeme do mísy, přidáme vanilkový cukr, nakrájené máslo a zpracujeme na drobenku. Polovinu dáme stranou na posypání, do zbylé přidáme žloutek a propracujeme. Hotovou drobenku namačkáme na dno do kulaté formy.

Do povařeného pudinku zašleháme mletý mák. Dvě třetiny mléka saříme s cukrem. Zbylé mléko smícháme s pudinkovým práškem a vlijeme do vroucího mléka s cukrem. Povaříme přibližně po dobu dvou minut. Poté přidáme mletý mák, vše zamícháme a odstavíme. Směs dáme stranou a necháme vychladnout.

Do makové náplně lehce vmícháme bílkový sníh. Z bílku důkladně vyšleháme pevný sníh. Ten pak opatrně vmícháme do připravené vychladlé makové náplně. Hotovou náplň lehce rozetřeme do formy na namačkanou drobenku. Mezitím zapneme troubu a předehřejeme ji na 160\degree C.

Na mandarinky nasypeme zbytek drobenky. Na makovou náplň ozdobně naskládáme osušené mandarinky. Nakonec povrch posypeme zbylou drobenkou. Dáme do trouby a pečeme asi 50 minut. Poté necháme krátce vychladnout ve formě. Pak formu sejmeme a koláč nakrájíme.
\end{minipage}

\subsection{Lívance}
\begin{minipage}[t]{0,5\textwidth}
\begin{myitemize} 
\item máslo na smažení
\item špetka soli
\item 300g hl mouky
\item 2 vejce
\item prdopeč
\item 400ml mléka
\item 6 lžic oleje
\item 4 lžíce cukru
\end{myitemize}
\end{minipage}
\begin{minipage}[t]{0,5\textwidth}
Suroviny kromě másla a prdopeče míchat několik minut do hladkého těsta. Nechat 30 min odestát. Vmíchat prdopeč a nechat chvilku bobtnat. Smažit na trošce másla na mírném plameni.
\end{minipage}

\subsection{Tvarohový táč meruňkový}
\begin{minipage}[t]{0,5\textwidth}
\textbf{Našlehat}
\begin{myitemize} 
\item 1kg tvarohu
\item 4 vejce
\item 125g rozpuštěného másla
\end{myitemize}
\textbf{Směs}
\begin{myitemize} 
\item 4 lžíce krupičky
\item 7 lžic cukru
\item 1 prdopeč
\item 1 vanilkový cukr
\end{myitemize}

\textbf{Ozdoba}
\begin{myitemize} 
\item krájené ovoce
\item drobenka
\end{myitemize}

\end{minipage}
\begin{minipage}[t]{0,5\textwidth}
Našlehané tvarohy smíchat se směsí. Možno přidat hrozinky. Těsto rozetřít na plech, poklást ovocem a posypat drobenkou. Péct 45 minut. 
\end{minipage}

\pagebreak
\section{Polívky}
\subsection{Bramborová s uzenou rybou}
\begin{minipage}[t]{0,5\textwidth}
\begin{myitemize} 
\item 25g másla
\item 1/2 cibule
\item 200g brambor
\item 400ml vývaru
\item 100ml mléka
\item 200ml smetany ke šlehání
\item 200g uzené ryby
\item 100g pancetty
\item javorový sirup
\item pažitka
\item čerstvý mletý pepř
\end{myitemize}
\end{minipage}
\begin{minipage}[t]{0,5\textwidth}
Cibulku osmahneme,potom přidáme brambory a opečeme je. Zalijeme vývarem a přivedeme k varu. Necháme vařit, dokud nezměknou (cca 15min). V dalším rendlíku přivedeme k varu mléko se smetanou. Vložíme rybu a 3 minuty prohřejeme. Brambory ve vývaru rozmixujeme. Přidáme mléko se smetanou (rybu vyjmeme), opepříme. Pancettu opečeme v pánvi, dáme ji do misek spolu s rybou a zalijeme polévkou. Pokapeme javorovým sirupem, posypeme pažitkou.
\end{minipage}

\subsection{Rybí polévka}
\begin{minipage}[t]{0,5\textwidth}
\begin{myitemize} 
\item Kapr
\item Kořenová zelenina
\end{myitemize}
\end{minipage}
\begin{minipage}[t]{0,5\textwidth}
Do velkého hrnce dáme vařit vnitřnosti, do menšího máslovou zásmažku (já ji dělám na pánvičce).

Do prostředního kořenovou zeleninu, kterou (kostičky) opražíme na másle do zlatova, osolíme, zalijeme vodou a přidáme hlavy. Vaříme asi půl hodiny, pak hlavy vyndáme a po vychladnutí důkladně obereme. 

Do hrnce se zeleninou přidáme zásmažku, provaříme a vše přecedíme do hrnce s vnitřnostmi. 

Nakonec přidáme maso z hlav a dochutíme. 
\end{minipage}

\subsection{Batátový krém}
\begin{minipage}[t]{0,5\textwidth}
\begin{myitemize} 
\item 1 střední sladká brambora (cca 300 g)
\item 2 stroužky česneku
\item 1 cibule
\item 1 lžíce olivového oleje
\item 1 lžička uzené papriky
\item sůl
\item 300 ml vývaru
\item 100 ml plnotučného mléka (smetany)
\item citrónová šťáva
\item bílý jogurt k podávání
\end{myitemize}
\end{minipage}
\begin{minipage}[t]{0,5\textwidth}
Bramboru a cibuli jsem si oloupala a nakrájela na menší kostky. Česnek jsem nechala tak se slupkou. Promíchala jsem olivový olej se solí a uzenou paprikou. Směs jsem přelila na brambory s cibulí, česnekem a důkladně promíchala, aby se každý kousek obalil. Dala jsem péct do předehřáté trouby na 30-40 minut. Během pečení jsem párkrát promíchala. Jakmile byly brambory měkké, vytáhla jsem z trouby, česnek jsem vymačkala a promíchala se směsí. Vše jsem přesunula do hrnce a zalila vodou. Dala jsem vařit asi na 10 minut. Na závěr jsem rozmixovala, přilila mléko a ještě chvíli povařila. Dosolila jsem a dochutila citrónovou šťávou. Rozdělila jsem na talíře, ozdobila lžičkou bílého jogurtu a jemně poprášila uzenou paprikou.
\end{minipage}

\subsection{Ramen}
\begin{minipage}[t]{0,5\textwidth}
\begin{myitemize} 
\item bůček
\item kus hovězího na polévku
\item pórek
\item nějaké tmavé asijské houby
\item řasu
\item asijské nudle do polévky
\item vajíčka podle počtu porcí
\item čerstvý zázvor
\item sójovku
\item rybí omáčku
\item čerstvé chilli
\item česnek
\item cibule
\item kořenová zelenina
\end{myitemize}
\end{minipage}
\begin{minipage}[t]{0,5\textwidth}
Do pekáče hodíme maso, kořenovou zeleninu, cibuli na půlky. Pečeme na 180 odkryté bez podlévání cca 2 hodiny, maso i zelenina by mělo ztmavnout, ne zčernat, pokud by to hrozilo, otočit, nebo stáhnout teplotu.

Po upečení vezmeme děrovanou lžíci, opatrně vložíme obsah pekáče do velkého hrnce a zalijeme větším množstvím vody. Přidáme zázvor a houby. Dáme na co nejmenší oheň a necháme táhnout min 6 hodin.
Z hrnce vybereme tuhý obsah, maso a zeleninu dáme zvlášť. Do vývaru přidáme nasekané chilli, sójovku, rybí omáčku, česnek na plátky, řasu. Povaříme až česnek změkne. Pórek nakrájíme a prudce povaříme ve vývaru.

Vejce uvaříme na hniličku nebo slow poached \ref{slowpoached}. Vezmeme uvařený bůček a nakrájíme ho na plátky, nebo ho natrháme, poté ho osmažíme na pánvi do zlatova. Mezitím uděláme nudle dle návodu.

Do velké misky dáme nudle, zalejeme vývarem, přidáme maso a nakonec vejce. Sežereme.
\end{minipage}


\pagebreak
\section{Udírna}
\subsection{Grilovaná žebírka z Oklahomy}
\begin{minipage}[t]{0,5\textwidth}
\begin{myitemize} 
\item žebra-bůček?
\item cukr tmavý
\item pílý pepř, černý pepř - hodně
\item cibulka najemno
\item sladká paprika
\item chilli
\item římský kmín
\item sůl
\end{myitemize}
\end{minipage}
\begin{minipage}[t]{0,5\textwidth}
koření a cibuli strčit do mixéru a najemno

nechat přes noc naležet a následne udit při nízké teplotě 5-6 hodin
\end{minipage}
\subsection{Trhané veřové}
\begin{minipage}[t]{0,5\textwidth}
\begin{myitemize} 
\item Teploměr do masa
\item Silný alobal
\item Krkovice s kostí - co největší kus, klidně 3,4 kila, hodně tučný
\item Cider, ten anglický, nedoslazovaný
\item Jablečný juice
\item Sušená, granulovaná cibule
\item Sušený, granulovaný česnek
\item Sladkáý paprika
\item Pepř vícebarevný, nebo černý, nahrubo namletý
\item Hrubozrnná sůl
\item Cibule
\end{myitemize}
\end{minipage}
\begin{minipage}[t]{0,5\textwidth}
Cibuli nakrájíme na půlkolečka a naplácáme na maso. Maso obalené v cibuli vložíme do sáčku a zalijeme ciderem, zabalíme a necháme den odležet. 

Připravíme si směs na obalení. Smícháme granulovaný česnek, granulovanou cibuli s pepřem, paprikou a solí. 

Maso vyndáme z pytle, položíme na rošt, zbavíme cibule a necháme chvíli větrat. Odloženou cibuli i ciredový lák dáme do pekáče (měl by být širší jak celé maso). 

Maso hustě pokryjeme kořenící směsí, a i s roštem umístníme na pekáč. Vpíchneme teploměr dostředu masa a umístníme do udírny vyhřáté na 100 \degree C. Pečeme do té doby než maso dosáhne teploty 70 \degree C, mělo bz to trvat 5 až 7 hodin podle velikosti masa. Celou dobu doplňujeme do pekáče vodu smíchanou s jablečným džusem, pozor ale aby to nebylo moc sladké.

Následně se maso vytáhne z udírny, položí do alobalu a polije džusem. Pekáč se dá bokem a do udírny už nepřijde. Poté se pečlivě zabalí do alobalu aby nemohla unikat šťáva. Takhle zabalené maso vrátit do udírny, mírně zvýšit teplotu a péct do té doby, než teplota masa bude 90 \degree C. 
Po vytažení se maso rozbalí a dá do nějaké mísy. Teď by maso mělo skoro samo odpadat od kosti. Nechat odležet alespoň půl hodiny. Šťávu z masa smíchejte s obsahem pekáče a zredukujte v omáčku. Maso natrhejte vidličkou. Sežrat!
\end{minipage}
\pagebreak
\section{Vepřík}
\subsection{Pečené uzené koleno}
\begin{minipage}[t]{0,5\textwidth}
\textbf{Na vývar}
\begin{myitemize} 
\item 1 ks uzeného kolena
\item  2 cibule
\item  větvičky čerstvého  rozmarýnu
\item  pepř dlouhý tři šišky
\item  sůl
\item  3 stroužky česneku
\end{myitemize}
\textbf{Marináda}
\begin{myitemize} 
\item 3 lžíce  medu, 
\item 1 lžíce worcesteru, 
\item 3 lžíce sójové omáčky, 
\item 2 lžíce slunečnicového  oleje, 
\item 3 lžíce koření grilovacího nebo dle chuti.
\end{myitemize}
\end{minipage}
\begin{minipage}[t]{0,5\textwidth}

Koleno opláchneme, očistíme. Tenkým nožem uděláme tři krátké řezy až ke kosti, aby se koleno lépe provařilo. Dáme vařit na 15 minut a pak vodu vyměníme. Odstraníme tím trochu chemikálií. Do vyměněné vody dáme větvičky rozmarýnu, "šišky" pepře, cibuli a asi jednu malou lžíci soli. Kolínko vaříme 50 minut. Uvařené koleno vyndáme a kůži na povrchu asi 0.5 cm hluboko nařízneme na mřížku. Oloupané stroužky česneku rozpůlíme a vtlačíme na několika místech do kolene. Pomažeme marinádou asi dávkou 1/3, podlijeme třemi lžícemi vývaru z kolena a dáme do trouby. Pečeme v troubě asi 30 minut na 200 stupňů. Průběžně přidáváme marinádu a poléváme troškou vývaru z kolena.
\end{minipage}

\subsection{Pečené fazole na pivě}
\begin{minipage}[t]{0,5\textwidth}
\begin{myitemize} 
\item 200 g anglické slaniny, na kostičky
\item 4 cibule, nakrájené nahrubo
\item 3 stroužky česneku, drcené
\item 800g bílých fazolí ve slaném nálevu, scezených
\item 200 ml tmavého piva 
\item 1,5 lžíce jablečného octa
\item 1,5 lžíce třtinového cukru
\item 4 lžíce hustého rajského protlaku
\item 1 lžíce dijonské hořčice
\item 1,5 lžičky mletého chilli
\item 1 lžička hrubé soli
\item 1 lžička mletého pepře
\item chleba k podávání
\end{myitemize}
\end{minipage}
\begin{minipage}[t]{0,5\textwidth}
Troubu předehřejte na 200 \degree C. V kastrolu nebo v pánvi s ohnivzdornou rukojetí opékejte na středním plameni slaninu. Vyjměte děrovanou sběračkou a dejte okapat na papírové utěrky. Do nádoby přidejte olej, cibuli a opékejte za stálého míchání 8 minut dorůžova. Přidejte česnek a míchejte další minutu. Přidejte fazole, pivo, ocet, melasu, protlak, hořčici, chilli, sůl a pepř, promíchejte, vmíchejte slaninu a přiveďte k varu. Přendejte do trouby a pečte do zhoustnutí zhruba 30–40 minut. Podávejte s chlebem.
\end{minipage}

\subsection{Vepřové výpečky}
\begin{minipage}[t]{0,5\textwidth}
\begin{myitemize} 
\item 1 kg libového bůčku i s kůží (plecka)
\item 2 střední cibule
\item 3 větší stroužky česneku
\item 1 KL celého kmínu
\item sádlo
\item pepř, sůl (15g)
\end{myitemize}
\end{minipage}
\begin{minipage}[t]{0,5\textwidth}
Maso nakrájíme na větší kostky (4x4x4cm), prohněteme se solí a odložíme. V kastrolu na sádle zpěníme nadrobno nakrájenou cibuli s česnekem na plátky a kmínem. Vložíme prosolené maso a orestujeme po všech stranách až maso zbledne (hrůzou). Po té jej podlijeme hrnkem vřící vody, vrazíme to přikryté do trouby rozpálené na 200 \degree C a pečeme 1,5 hodiny s občasným promícháním a dolitím vody. Po uvedené době sejmeme poklici a pečeme ještě asi 1/2 hodiny odkryté bez podlívání. Na závěr dolijeme vřící vodu na potřebné množství šťávy, setřeme ze stěn výpek a rozmícháme ve šťávě (můžeme ještě provonět jedním utřeným stroužkem česneku). Podáváme s brambor. knedlíkem a špenátem.
\end{minipage}

\subsection{Vepř na pivu a kávě s cibulkama}
\begin{minipage}[t]{0,5\textwidth}
\begin{myitemize} 
\item Vepřové, tučnější (ocásky, krkovice, bůček)
\item \textbf{Nálev na 1kg masa}
\item 130g soli
\item 75g cukru
\item 1.5 litru vody
\item \textbf{Marináda}
\item česnek 
\item hodně protlaku a trošku hořčice
\item worchester
\item čerstvá metá káva
\item povidla
\item pivo
\item med
\item \textbf{Na cibulky}
\item cukr
\item malé nakládané cibulky
\item balzamikový ocet
\end{myitemize}
\end{minipage}
\begin{minipage}[t]{0,5\textwidth}
Nakládáme v nálevu půl dne až den, potom celý den v marinádě, kterou připravíme  smícháním všech surovin a rozšleháním v mixéru.
Pečeme v troubě asi na 110 \degree C 3 a více hodin podle velikosti masa. 

V pánvi rozpustíme cukr, přidáme cibulky, promícháme a přidáme ocet, necháme vyvařit.
\end{minipage}

\subsection{Pečené buřty na pivě}
\begin{minipage}[t]{0,5\textwidth}
\begin{myitemize} 
\item nějaké měkčí klobásy, kvalitní špekáčky
\item cibule
\item rajčata (plechovka, čerstvé...)
\item protlak
\item horčice
\item paprika, kapie
\item chilli
\item jablečný ocet
\item stout
\end{myitemize}
\end{minipage}
\begin{minipage}[t]{0,5\textwidth}
Rajčata povaříme s horčicí, chilli, octem. Poté přidáme protlak a vodu. Směsi by mělo být cca stejně jako piva. 

Do hlubokého pekče nakrájíme na kolečka cibuli aby pokryla vě větši vrstvě dno. Na to naskládáme buřty pokrájené na kousky. Na to naskládáme nakrájenou papriku. Zalijeme rajčatovou směsí a pivem. Vrazit do trouby a péct do zhoustnutí.
\end{minipage}
\pagebreak
\section{Těstoviny}
\subsection{Lasagne alá Terezka}
\begin{minipage}[t]{0,5\textwidth}
\begin{myitemize} 
\item 100g listový špenát
\item cca 500-750g masová hovězí směs
\item bešamel
\item plátky lasagní
\item parmezán na posypání
\end{myitemize}
\textbf{Bešamel}
\begin{myitemize} 
\item 4 lžíce hladké mouky
\item 40g másla
\item 600ml mléka
\item 50g parmezánu
\item sůl, pepř
\end{myitemize}
\textbf{Masová směs}
\begin{myitemize} 
\item 600g hovězího mletého masa
\item 4 plátky slaniny
\item velká cibule
\item česnek 
\item bobkový list
\item směs bylinek (rozmarýn, tymián, bazalka..), římský kmín
\item protlak
\item konzerva kráj. rajčat
\item 200ml hovězího vývaru
\item sůl, pepř
\end{myitemize}
\end{minipage}
\begin{minipage}[t]{0,5\textwidth}
Máslo necháme na pánvi rozpustit, přidáme hl. mouku a mícháme asi minutu. Vešleháme studené mléko a přivedeme k varu. Provaříme 5 minut. Osolíme, opepříme, přidáme špetku muškátového květu. Vypneme po pěti minutách a odstavíme, nakonec vmícháme parmezán.

Na oleji zprudka opečeme slaninu nakrájenou na nudličky. Přidáme cibuli, česnek a bobkový list a ztlumíme oheň a opékáme asi 5 minut. Směs vyjmeme a do pánve vložíme maso, které opečeme dohněda. Po asi 6-8mi minutách přidáme k masu předtím vyndanou směs a dále přidáme protlak, bylinky+kmín a červené víno. Podusíme, aby se víno zredukovalo na polovinu. Pak přidáme hovězí vývar a plechovku krájených rajčat. Osolíme, opepříme, zvolna vaříme 30-40 minut do zhoustnutí.

K masové směsi přimícháme spařený špenát (můžeme, nemusíme). Dno zapékací mísy potřeme polovinou masové směsi. Překryjeme plátky lasagní. Na to rozetřeme polovinu bešamelu. Na to nalijeme zbytek masové směsi, zase poklademe plátky lasagní a nakonec potřeme zbytkem bešamelu. Posypeme parmezánem a dáme zapéct na 30 minut na 200 stupňů. Nakonec 5 minut nastavit troubu na gril nezaškodí :)
\end{minipage}
\pagebreak
\section{Hovězí}
\subsection{Chilli con carne}
\begin{minipage}[t]{0,5\textwidth}
\begin{myitemize} 
\item lžíce oleje
\item velká cibule nadrobno
\item červená paprika na kostičky
\item 2 stroučky česneku
\item chilli 
\item sladká aprika
\item římský kmín
\item 500 mleté hovězí
\item 300 hov. vývar
\item 400 konzerva rajčat
\item 1\textbackslash 2 lžíce sušené majoránky
\item 2 ližičky protlaku
\item 400 červených fazolí
\item sůl pepř
\item kostka hořké čokolády
\end{myitemize}
\end{minipage}
\begin{minipage}[t]{0,5\textwidth}

\end{minipage}
\subsection{Pomalu pečené hovězí se švestkami}
\begin{minipage}[t]{0,5\textwidth}
\begin{myitemize} 
\item Kilo a více hovězí kližky, nebo zvěřiny
\item česnek
\item bobkový list
\item čerstvý tymián
\item sušené houby
\item cca dvě piva (tmavé, polotmavé, ne příliš hořké)
\item portské 300 ml
\item 200g sušených švestek bez pecky
\item 2 lžíce balzamikového octa
\item máslo, olej
\item 30 dkg anglické v celku
\item alespoň půl kila šalotek v celku
\item 4 řapíky celeru
\item půl kila malých žampionů
\item hladká mouka
\item  sůl pepř
\end{myitemize}
\end{minipage}
\begin{minipage}[t]{0,5\textwidth}
\textbf{Co udělat den až dva dopředu}\\

1. Maso vložte do keramické nebo skleněné mísy, posypte česnekem, bobkovými listy, tymiánem a sušenými houbami, pokud jste se rozhodli je použít. Vlijte pivo a 200 ml portského. Přetáhněte fólií nebo jinak zakryjte a nechte přes noc nebo podle potřeby marinovat.

2. Švestky vložte do misky a přelijte balzamikovým octem a zbývajícím portským. Přiklopte nebo přetáhněte fólií a nechte přes noc v chladnější místnosti.\\

\textbf{V den D (pečení)}\\

3. Zahřejte troubu na 150 \degree C, horkovzdušnou na 130 \degree C. Maso slijte, marinádu přitom uschovejte. Maso osušte. Máslo a olej zahřejte v hrnci vhodném do trouby. Na vysokém plameni opečte po částech maso ze všech stran do tmavohněda. Maso vyjměte z kastrolu a uchovejte v teple. Pokud by se tuk přepaloval, vylijte ho, kastrol vytřete kuchyňským papírem a přidejte další trochu oleje a másla.

4. Do hrnce přidejte slaninu, šalotky a celer a opečte dozlatova. Vmíchejte houby a za stálého míchání je 2–3 minuty opékejte. Vmíchejte mouku, pak přidejte maso a uschovanou marinádu. 

5. Přiveďte k ustálenému varu. Přiklopte dobře těsnícím víkem vhodným do trouby a vložte na 2 a půl až 3 hodiny do trouby, aby maso opravdu pořádně změklo.

6. Ochutnejte, případně osolte. Přidejte švestky i s nálevem a duste ještě 30 minut.
\end{minipage}

\subsection{Pomalu pečené hovězí na balzamiku}
\begin{minipage}[t]{0,5\textwidth}
\begin{myitemize} 
\item 2Kg hovězího na pečení
\item sůl 24g
\item olivový olej
\item cibule
\item česnek
\item bobkový list
\item hovězí vývar
\item balzamikový ocet (125ml)
\item rajčatový protlak
\item worchester
\item med
\item dijonská hořčice
\item čerstvý rozmarýn
\item bramborová kaše a hrášek k podávání
\end{myitemize}
\end{minipage}
\begin{minipage}[t]{0,5\textwidth}
Pečeme na 160 \degree C. Z masa uděláme hezký balíček, svážeme pokud je potřeba, osolíme, opepříme a opečeme ze všech stran a vytáhneme z kastrolu.

Opečeme cibulku, posléze česnek a bobkový list. Přidáme zbytek surovin a maso. Pečeme cca 4 hodiny. 

Po upečení vyndáme maso a bobkový list. Omáčku zahustíme. Podáváme s kaší a hráškem prohřátým na másle.
\end{minipage}

\subsection{Hovězí po Brugundsku}
\begin{minipage}[t]{0,5\textwidth}
\begin{myitemize} 
\item Hovězí (zadní)
\item Žampiony
\item Máslo
\item Rozmarýn
\item Hovězí vývar
\item Červené víno
\item Protlak
\item Česnek
\item Cibule
\item Celer
\item Mrkev
\item Anglická slanina v celku
\end{myitemize}
\end{minipage}
\begin{minipage}[t]{0,5\textwidth}
Maso nakrájet na kostky, osolit opepřit a zprudka osmažit. Vybrat z hrnce. Osmažit slaninu a vyndat z hrnce. Osmažit cibuli a potom přidat zeleninu, poté česnek a protlak a ještě chvíli nechat smažit. Přidat víno a trochu zredukovat, přidat koření, maso a slaninu. Péct přikryté dvě až tři hodiny 160\degree - 180\degree
\end{minipage}

\subsection{Hovězí guláš na pivě}
\begin{minipage}[t]{0,5\textwidth}
\begin{myitemize} 
\item Hovězí kližka 1kg
\item Cibule 1kg \textit{Mělo by jí být vizuálně dvojnásobek masa, čím víc cibule, tím sladší guláš}
\item Protlak
\item Worchestrová omáčka l\textit{žádnej albert shit}
\item Černé pivo 0,5l \emph{kvalitní, bez karamelu, ideálně nějakej stout, ne moc hořké}
\item Sladkou papriku \emph{opět kvalitní, ideálně pravou maďarskou} 
\item Kmín
\item Majoránka
\item Česnek
\item Chilli \textit{dle chuti, dávám dvě čerstý červený}
\item sádlo
\item sůl
\end{myitemize}
\end{minipage}
\begin{minipage}[t]{0,5\textwidth}
Nakrájíš cibuli nahrubo, osolíš a dáš smažit do většího hrce s rovným dnem. Ideálně aby \textit{neměl} nějakou nepřilnavou vrstvu, měly by se dělat příškvarky. Cibuli smažit do té doby než lítá po hrnci a je hodně tmavá. Trvá to hodně dlouho tak přeji pevné nervy. Je to důležitá část, pokud se to přežene se sádlem, dá se potom po osmažení cibule odebrat. Na začátku smažit na velkém ohni a ani se nemusí tolik míchat, ale čím dál víc to bude tmavnout, tím víc je potřeba dávat si bacha. Brát hodně ode dna a odškrabovat příškvarky.

Až je cibule osmažená, přidát kmín, papriku, a kližku nakrájenou na větší kostky. Z masa vykrájet jen opravdu velké kusy kližky, nebo vůbec nevykrajovat. Maso prudce zatáhnout a zalít pivem. Bacha na připálení papriky a cibule. Dolít horkou vodou aby maso plavalo. Přidat zbytek koření. Guláš by měl začít houstnout. Vařit 2+ hodiny a dolévat vodu.

Když chceš guláš v sobotu, začni ho dělat ve středu a každý den ho povař, dolij vodou.

Hack na závěr. Pokud chcete dosáhnout té chuti a hnědočerné barvy jakou vidíte sem tam v hospodě, tak jejich tajemství není totálně spálená cibule, ale kulér! Stačí lžička - je to takové sladkokyselé.


\end{minipage}
\subsection{Maďarský Perkelt}
\begin{minipage}[t]{0,5\textwidth}
\begin{myitemize} 
\item Sádlo
\item 3/4 kila cibule
\item kilo hovězího (hližka, pečeně)
\item mletá sladká paprika
\item zelená paprika
\item chilli
\item 4 velká rajčata
\item hovězí vývar
\end{myitemize}
\end{minipage}
\begin{minipage}[t]{0,5\textwidth}
1 Na sádle opékejte cibuli zvolna za stálého míchání asi 10 minut dorůžova. Měla by hodně změknout, ale ne se připálit. Zasypte ji paprikou a zamíchejte. Ihned přidejte maso, promíchejte a zvyšte plamen.

2 Když maso zesvětlá a zatáhne se, vsypte papriky, rajčata a chilli a přiveďte k varu. Opět ztlumte a duste přiklopené ve vlastní šťávě doměkka, trvá to nejméně 2 hodiny.

3 B\v{e}hem va\v{r}en\'{\i} podl\'{e}vejte, ale jen m\'{a}lo \textendash{} tak, aby maso bylo t\'{e}m\v{e}\v{r} pono\v{r}en\'{e}, ale neplavalo. Ma\v{d}arsky\' perkelt ani gul\'{a}\v{s} se nezahu\v{s}\v{t}uj\'{\i}, tak\v{z}e si p\v{r}i dokon\v{c}en\'{\i} na\v{r}e\v{d}te tekutinu, jak chcete (perkelt husty\', gul\'{a}\v{s} \v{r}\'{\i}dky\'), a podle chuti osolte a opep\v{r}ete. Pod\'{a}vejte v misce s chlebem, v Ma\v{d}arsku se n\v{e}kdy j\'{\i} se speci\'{a}ln\'{\i}mi t\v{e}stovinami zvany\'mi tarho\v{n}a.
\end{minipage}

\subsection{Svíčková}
\begin{minipage}[t]{0,5\textwidth}
\begin{myitemize} 
\item 6 velkých mrkví
\item 1 celer
\item 6 petrželí
\item 1.5 kila hovězího
\item 200 slaniny
\item 5 bobkových listů
\item 8 kuliček nového koření
\item 8 kuliček pepře
\item sůl
\item smetanu ke šlehání
\item mouka podle potřeby 
\end{myitemize}
\end{minipage}
\begin{minipage}[t]{0,5\textwidth}
1. Připravte si zeleninu
Kořenovou zeleninu očistěte a nastrouhejte na hrubém struhadle, cibuli oloupejte a pokrájejte na kostičky. Rozpusťte máslo a odblaňte maso, pokud je to potřeba.

2. Dopřejte masu odpočinek
Hovězí důkladně prošpikujte slaninou tak, aby pokud možno nevyčnívala. Do většího pekáče vložte třetinu zeleniny, prošpikované maso a koření a zalijte je rozpuštěným máslem. Pak hovězí zasypte zbytkem zeleniny a nechte nejméně přes noc v chladu.

3. Upečte hovězí doměkka
Rozpalte troubu na 160 \degree C. Dejte do ní maso se zeleninou a pečte asi deset minut bez podlévání. Pak vše osolte, podlijte vroucí vodou
nebo vývarem a za občasného podlévání pečte doměkka. Poté maso s kořením vyjměte a zeleninu dohladka rozmixujte.

4. Dokončete omáčku
Mouku rozmíchejte ve šlehačce s mlékem (můžete klidně použít jen mléko; výsledek pak nebude tak tučný), a pokud je to třeba, omáčku zahustěte. Ještě chvíli povařte. Dosolte a případně dotáhněte kysanou smetanou. Omáčku můžete také mírně přisladit.

Podávejte ozdobené nakládanými brusinkami.

Tipy a triky
Pro přípravu svíčkové je samozřejmě nejvhodnější hovězí maso (nejšťavnatější je z jatečného býka), nejlíp v kvalitě bio. S úspěchem
však využijete i maso králičí a maso ze zajíce (pokud ho seženete).

Prošpikované a naložené maso nechte odpočívat nejméně 1 den (nejlépe 2–3 dny) v lednici. Máslo po vychlazení ztuhne a zamezí
přístupu vzduchu. Maso bude měkčí a lahodnější.

Hovězí raději solte až během pečení, jinak by se z něj zbytečně vyluhovalo mnoho živin.

Svíčkovou můžete také naložit do mořidla. Pět minut povařte směs vody a octa (po 400 ml), soli a pokrájené nebo nastrouhané kořenové zeleniny. Vychladlým mořidlem pak přelijte odblaněné a prošpikované maso a nechte nejméně 1 den odležet. Maso krásně zkřehne.
\end{minipage}

\subsection{Švédská roštěná}
\begin{minipage}[t]{0,5\textwidth}
\begin{myitemize} 
\item 600g nízkého roštěnce
\item sardelová pasta
\item hladká mouka
\item cibule
\item slanina/anglická slanina
\item sádlo
\item voda/vývar na podlití
\item plnotučná hořčice
\item 50ml láku z okurek
\item 50g - menší nakládaná okurka
\item šlehačka
\item 50g kapary
\item černý pepř
\end{myitemize}
\end{minipage}
\begin{minipage}[t]{0,5\textwidth}
Maso rozdělit na plátky, cca 4, rukou naklepat, naříznout okraje, potřít sardelovou pastou, opepřit a poprášit moukou.

Cibuli nakrájet nadrobno a slaninu na kostičky. V hrnci rozpálit sádlo a maso opražit dozlatova, vyndat a do hrnce dát cibuli a slaninu - orestovat. Roštěnky vrátit zpět podlít vodou nebo vývarem, přidat lák smíchaný s hořčicí. Maso by mělo být lehce ponořené. Přiklopit a dusit doměkka 1,1-5 hodiny. Podle potřeby podlévat.

Okurku nakrájet na kostičky. Po změknutí masa podle potřeby vyvařit šťávu, přidat smetanu, okurky, kapary a ještě chvíli povařit.
\end{minipage}

\subsection{Beef Jerky}
\begin{minipage}[t]{0,5\textwidth}
\begin{myitemize} 
\item 1Kg zadního hovězího, co nejmíň tučného
\item 3/4 malého šálku Worcesterové omáčky
\item 3/4 malého šálku sojovky
\item lžíci uzené papriky
\item lžíci medu
\item 2 lžičky čerstvě mletého pepře
\item jednu chilli papričku, nebo dle chuti
\item lžičku sušeného česneku
\item lžičku sušené cibule
\end{myitemize}
\end{minipage}
\begin{minipage}[t]{0,5\textwidth}
Maso okrájíme od tuku. Dáme dvě hodiny na mražák, poté nakrájíme na plátky tlusté 3-5mm. Ze zbylých ingrediencí uděláme marinádu, maso do ní naložíme na 3 a více hodin (klidně přes noc). 

Troubu předehřejeme na 70\degree C. Plátky pověsíme do trouby na mřížku jako prádlo na šňůru, pod něj dáme plech na odkapávání. Sušíme 3-4 hodiny.
\end{minipage}

\pagebreak
\section{Kuřecí}
\subsection{Kuřecí stehna s medouvou hořčicí}
\begin{minipage}[t]{0,5\textwidth}
\begin{myitemize} 
\item hodně česneku
\item med
\item dijonská hořčice
\item sójovka
\item citronová šťáva
\item pepř
\end{myitemize}
\end{minipage}
\begin{minipage}[t]{0,5\textwidth}

\end{minipage}
\subsection{Kuře s harissou a pečenými paprikami}
\begin{minipage}[t]{0,5\textwidth}
\begin{myitemize} 
\item 500g kuřecích stehen
\item 2 lžíce harissy
\item 1 lžíce vinného č. octa
\item balkánský sýr
\item cibule na čtvrtky
\item pečené papriky
\end{myitemize}
\end{minipage}
\begin{minipage}[t]{0,5\textwidth}
\textbf{Pečené papriky}

Papriky pečeme celé v troubě při 200\degree C dokud nezčernají, potom dáme do mikroten. sáčku,aby se zapařily a pak je oloupeme a zbavíme semínek; zalijeme olivovým olejem a smícháme s česnekem a kořením podle chuti.

V pekáčku smícháme harissu s vinným octem a obalíme v této směsi kuřecí vykostěná stehna bez kůže a cibuli. Osolíme a dáme péct na 200\degree C na cca 35min. V průběhu pečení promícháme. 10min před koncem přidáme papriky a balkánský sýr, opepříme.
\end{minipage}
\subsection{Pomalu pečená kachna}
\subsubsection*{1. Osolte a okmínujte}
 Kachnu omyjte a osušte. Pak odřízněte přebytečný tuk u biskupa a u krku a kachnu důkladně osolte. Solte uvnitř i zvenku a snažte se sůl rozetřít do všech záhybů. Pak ji posypte kmínem, vložte do pekáčku, mírně podlijte vodou, přiklopte a pečte sedm hodin při 90 \degree C.

\subsubsection*{2. Zvyšte teplotu}
 Po sedmi hodinách bude vaše kachna vypadat, jako kdybyste ji jen lehce povařili, kůže bude mít žlutou barvu a maso ještě nebude úplně měkké (vidlicí nepropichujte). Zvyšte teplotu na 120 \degree C a dopřejte masu posledních šest hodin. Stále pečte přiklopené, nepodlévejte, tuk neodebírejte.

\subsubsection*{3. Pomažte medem }
 Hodinu před koncem celého pomalého pečení pekáč odklopte, do malého hrnečku si přelijte med a pomocí mašlovačky kachnu
 potřete; k medu můžete přidat i výpek. Kůže krásně zezlátne a bude křupavá. Podle uvážení můžete zvýšit teplotu trouby nebo dát kachnu pod gril.

\subsubsection*{4. Rozdělte prsa} 
 Kachnu naporcujte tak, jak jste zvyklí. Snažte se ale každému ze strávníků dát kousek prsíček, ta jsou po několikahodinové úpravě
 obzvlášť fenomenální! Maso je lahodné, krásně růžové, křupavé, a přitom vláčné. Podávejte s knedlíky a zelím.
\subsection{Paštika s brusinkami}
\begin{minipage}[t]{0,5\textwidth}
\begin{myitemize} 
\item 400 g kuřecích jater
\item 150 g másla + 50 g na závěrečné zalití
\item 1 stroužek česneku, prolisovaný
\item 1 šalotka, nasekaná
\item Bílé víno
\item Brusinky + brusinková marmeláda
\item 2 lžíce smetany ke šlehání
\end{myitemize}
\end{minipage}
\begin{minipage}[t]{0,5\textwidth}
Játra důkladně očistěte. Ve velké hlubší pánvi zahřejte oříšek másla a osmahněte na něm šalotku s česnekem, dvě minuty stačí. Nechte minutu rozvonět, přidejte játra a opečte je dohněda - při zmáčknutí by měla zůstat pružná, uvnitř růžová, ale zároveň ne syrová. Přidat trochu brusinek a půl lžíce marmelády, zacáknout vínem.

Promixujte na pastu. Přidat zbývající máslo, smetanu, sůl pepř a znovu promíchat.

Nandejte do sklenic a přelijte rozpuštěným máslem.
\end{minipage}

\subsection{Kuře s nádivkou}
\begin{minipage}[t]{0,5\textwidth}
\begin{myitemize} 
\item Kuře
\item Rohlík (1ks/porce)
\item Pórek
\item Mlíko
\item Kuřecí játra
\item Slanina
\item Cibulka
\item Máslo
\item Muškátový oříšek
\item Petrželka
\item Vajíčka
\end{myitemize}
\end{minipage}
\begin{minipage}[t]{0,5\textwidth}
Kuře opepříme, osolíme, naprudko zatáhmene a dáme péct do vysokého pekáče do trouby.\\
Osmažíme cibulku do hněda, přidáme slaninu, pórek. Přidáme na kousky nakrájená játra a částečně je uděláme, stáhneme z ohně.\\
Přidáme housky, mléko, vajíčka, sůl pepř a petrželku a naneseme na pečící se kuře.
\end{minipage}
\subsection{Kuře na houbách}
\begin{minipage}[t]{0,5\textwidth}
\begin{myitemize} 
\item 500g hříbků, nebo sušený namočený
\item 2 cibule
\item 2 stroučky česneku
\item olej
\item 4 kuřecí stehna
\item hladká mouka
\item dvě snítky tymiánu
\item dvě snítky rozmarýnu
\item 200ml suchého bílého vína
\item 2l kuřecího vývaru
\item 250ml smetany na vaření
\item 50g másla
\item sůl, pepř
\end{myitemize}
\end{minipage}
\begin{minipage}[t]{0,5\textwidth}
Hříbky nakrájet na plátky, cibuli nadrobno a česnek podrtit.

V hrnci co může do trouby, nebo pekáči s víkem rozehřát olej. Kuře osolit, opepřit a pomoučit. Kuře opéct ze všech stran dozlatova a vytáhnout z hrnce.

V hrnci opéct polovinu hub s cibulí, poté přidat bylinky, česnek lehce opražit a zalít vínem a vařit dokuď se tekutina neodpaří.

Do nádoby vrátit maso, zalít vývarem aby bylo maso ponořené a dusit 1h 15min v troubě předehřáté na 170 \degree C.

Po upečení vyndat maso z hrnce a šťávu zredukovat na třetinu. Poté přidat smetanu, ještě chvíli povařit a rozmixovat. Přidat kousek másla.

Zbylé hříbky orestovat na oleji se solí a vmíchat do omáčky.
\end{minipage}
\subsection{Máslové kuře s miso pastou a česnekem}
\begin{minipage}[t]{0,5\textwidth}
\begin{myitemize} 
\item kuřecí spodní stehna s kostí a kůží 8 ks
\item změklé máslo 4 lžíce
\item bílá miso pasta 1/2 hrnku
\item javorový sirup 3 lžíce
\item čerstvý zázvor, nastrouhaný 2cm kousek
\item česnek, neloupaný, ale lehce rozdrcený 8 stroužků
\item rýžový ocet 1 lžíce
\item čerstvě namletý černý pepř
\item jarní cibulka na ozdobu, nakrájená na tenká kolečka 2 ks
\item nasucho opražená sezamová semínka na ozdobu 2 lžičky
\end{myitemize}
\end{minipage}
\begin{minipage}[t]{0,5\textwidth}
Troubu předehřejte na 200 stupňů. Ve velké míse důkladně promíchejte změklé máslo, miso pastu, javorový sirup, rýžový ocet, nastrouhaný zázvor a černý pepř.

Kousky kuřete vložte do mísy a směs důkladně vmasírujte do masa ze všech stran. Prsty jemně nadzdvihněte kůži a pastu vetřete i pod kůži.

Kousky kuřete rozprostřete v jedné vrstvě do pekáčku, přidejte stroužky česneku a dejte do trouby. Pečte 30–40 minut, čas od času polévejte kůži vypečenou šťávou, až zezlátne a bude křupavá.

Kuře nandejte na talíře. Odstraňte slupku ze stroužků a upečený česnek rozmačkejte se šťávou v pekáči. Hotový sosík podlejte pod kuře. Pokrm ozdobte nakrájenou cibulkou a opraženými sezamovými semínky. Podávejte nejlépe s rýží.
\end{minipage}
\pagebreak
\section{Saláty}
\subsection{Grilovaná cherry rajčata s domácím pestem}
\begin{minipage}[t]{0,5\textwidth}
\begin{myitemize} 
\item Cherry rajčátka, nejlépe na větvičce
\item Parmezán (grand moravia)
\item čerstvý tymián a rozmarýn
\item Semínka - piniová nebo slunečnicová
\item Hodně česneku, jen oloupat
\item Olivový olej
\item Mořská sůl
\item Pepř
\end{myitemize}
\end{minipage}
\begin{minipage}[t]{0,5\textwidth}
V hmoždíři rozbijeme semínka, nakrájený rozmarýn, hrubou mořskou sůl, pepř a olivový olej. (Ten sůl a olej tam je důležitý, jinak ty semínka skoro nejdou rozbít)

Na grilovací plát nebo hoodně velkou a fakt rozpálenou pánev dáme rajčátka, tymián a česnek. Po chviličce přidat olivový olej. Až rajčátka popraskají tak to sundáme a rajčátka dáme do velké mísy, česnek parmezán a tymián do hmoždíře. Podrtíme a do mísy přidáme směsi z hmoždíře. Sníme...
\end{minipage}

\subsection{Kuře na paprice}
\begin{minipage}[t]{0,5\textwidth}
\begin{myitemize} 
\item Kuře
\item máslo
\item olej
\item cibule
\item máslová jíška
\item sladká paprika
\item mléko
\item šlehačka
\item vývar
\end{myitemize}
\end{minipage}
\begin{minipage}[t]{0,5\textwidth}

1. V rozpáleném kastrolu rozehřejeme máslo společně s olejem.

2. Maso důkladně osolíme, opepříme a přidáme do kastrolu, restujeme ze všech stran dozlatova, nejdříve kůží dolů (cca 10 minut).

3. Přidáme nadrobno nakrájenou cibuli a společně s masem restujeme, dokud cibulka nezrůžoví (cca 15 - 20 minut).

4. Opečené maso vyjmeme a do cibule přidáme mletou papriku. Krátce restujeme, stačí půl minuty. Maso vrátíme do kastrolu, zalijeme vodou, přidáme kostku bujónu a přivedeme k varu.

5. Přikryjeme pokličkou a snížíme teplotu. Kuře dusíme při mírné teplotě doměkka, cca 45 minut.

6. Měkké kuře vyjmeme z kastrolu a zahustíme jíškou.

7. Přidáme mléko a 15 minut provaříme při mírné teplotě.

8. Mezitím kuře vykostíme (pokud nepodáváme celou porci kuřete) a zbavíme kůže, nakrájíme na přiměřené kousky.

9. Do omáčky přilijeme smetanu, přivedeme k varu a stáhneme z plotny.

10. Omáčku scedíme přes husté síto, příp. můžeme rozmixovat v mixeru dohladka.

11. Do omáčky vložíme maso, necháme prohřát a podáváme s kynutými knedlíky, příp. těstovinami nebo rýží.
\end{minipage}

\pagebreak
\section{Omáčky, marinády a jiné}
\subsection{BBQ}
\begin{minipage}[t]{0,5\textwidth}
\begin{myitemize} 
\item Na lžíci oleje opéct cibulku a česnek nakrájené nadrobno se špetkou chilli do sklovata
\item Přidat 400g krájených rajčat
\item 80g třtinového cukru
\item 2 lžíce jablečného octa
\item 2 lžíce worcesterské omáčky
\item 1 lžíce rajského protlaku
\item Nechat bublat asi půl hodiny, rozmixovat
\end{myitemize}
\end{minipage}
\begin{minipage}[t]{0,5\textwidth}

\end{minipage}
\subsection{Smetanová omáčka s plísňovým sýrem}
\begin{minipage}[t]{0,5\textwidth}
\begin{myitemize} 
\item 6 posekaných jarních cibulek
\item 200g plísňového sýra
\item 300 zakysané smetany
\item wotchesterová omáčka
\item sůl, pepř
\end{myitemize}
\end{minipage}
\begin{minipage}[t]{0,5\textwidth}

\end{minipage}
\subsection{Jogurtová raita s okurkou}
\begin{minipage}[t]{0,5\textwidth}
\begin{myitemize} 
\item neoloupaná okurka bez pecek, nastrouhaná
\item 2 nakrájené jarní cibulky
\item strouček česneku
\item trošku čerstvého zázvoru
\item 2 lžíce čerstvé nasekané máty
\item 400 jogurtu
\item sůl pepř
\item lžička opraženého římského kmínu
\end{myitemize}
\end{minipage}
\begin{minipage}[t]{0,5\textwidth}

\end{minipage}


\subsection{Slow poached egg}
\label{slowpoached}
\begin{minipage}[t]{0,5\textwidth}
Hardcore verze vajec na hniličku
\begin{myitemize} 
\item vejce
\item teploměr
\item velký hrnec
\end{myitemize}
\end{minipage}
\begin{minipage}[t]{0,5\textwidth}
Do velkého hrnce dejte nějakou mřížku, napařovák nebo něco podobného. Vodu ohřát na 60 \degree C, vložit vejce a vařit 40-45 minut. Opatrně vyklepnout nad děrovanou lžící. Potom obdivovat nebo sežrat.
\end{minipage}

\subsection{Schezwan omáčka}
\begin{minipage}[t]{0,5\textwidth}
\begin{myitemize} 
\item Česneková pasta 2 lžíce
\item Zázvorová pasta 1 lžíce
\item Kmín celý
\item pasta z chilli, lžičku
\item červená chilli omáčka, lžička
\item ocet
\item cukr
\item olej
\item sůl, vodu
\end{myitemize}
\end{minipage}
\begin{minipage}[t]{0,5\textwidth}
Na pánvi rozpálit olej, dovnitř hodit česnek, kmín a zázvor, míchat.

Přihodit chilli omáčku sůl a dál míchat, potom přidat chilli a cukr a vařit cca 2 minuty. Přidat vodu a ocet a ještě trochu povařit.
\end{minipage}

\subsection{Utopenci}
\begin{minipage}[t]{0,5\textwidth}
\begin{myitemize} 
\item 8 špekáčků
\item cibule na kolečka
\item plnotučná hořčice
\item mletý pepř
\item nakládané kapie

\textbf{Nálev}
\item 200 octa
\item lžička cukru a lžička soli
\item 2 bobkové listy
\item 2 lžičky celého černého pepře
\item lžička nového koření
\item feferonka
\end{myitemize}
\end{minipage}
\begin{minipage}[t]{0,5\textwidth}
1 Přísady na nálev spolu s 500 ml vody povařte 2 minuty a nechte vychladnout.

2 Špekáčky oloupejte, podélně nakrojte, ale nedokrojte. Uvnitř řezu opepřete, potřete plnotučnou hořčicí a vyložte cibulí. Střídavě se zbylou cibulí a kapií vrstvěte do zavařovací lahve – měly by se vám vejít vždy dva špekáčky do jedné vrstvy.

3 Zalijte nálevem a nechte uležet v chladničce nejméně 10 dní.
\end{minipage}
\pagebreak
\section{Přílohy}
\subsection{Smetanové brambory s řapíkatým celerem}
\begin{minipage}[t]{0,5\textwidth}
\begin{myitemize} 
\item Bramory... cca kilo
\item dva kelímky sýru cottage nebo něčeho podobnýho
\item Jogurt
\item Řapíkatý celer
\item Dijónská hořčice 2 lžíce
\item Červený vinný ocet 1 lžíce
\item Olivový olej 1 lžíce
\item Pepř, sůl
\end{myitemize}
\end{minipage}
\begin{minipage}[t]{0,5\textwidth}
Uvař brambory do měkka, bez soli

Ve velké míse smíchej olej, hořčici, ocet, pepř a hodně soli - nejlépe mořská. Tuhle směs smíchej s čerstvě slitýma a horkýma bramborama. Počkej až vychladnou a přidej vyšlehaný jogurt se sýrem a řapíkatý celer nakrájený na kousky.

Před jídlem aspoň na hodinu do ledničky.
\end{minipage}
\subsection{Pečené brambory se smetanou a parmezánem}
\begin{minipage}[t]{0,5\textwidth}
\begin{myitemize} 
\item Brambory nastrouhané na plátky
\item parmezán
\item mléko
\item smetana
\item česnek
\end{myitemize}
\end{minipage}
\begin{minipage}[t]{0,5\textwidth}
Brambory naskládáme do misky a zaleijeme směsí z uvedených surovin... Pepř a sůl se taky hodí 
\end{minipage}
\subsection{Pečené brambory s rozmarýem}
\begin{minipage}[t]{0,5\textwidth}
\begin{myitemize} 
\item Brambory na plátky nebo na osminky
\item česnek 
\item sůl mořská
\item pepř hodně 
\item čerstvý rozmarýn
\item olivový ol.
\end{myitemize}
\end{minipage}
\begin{minipage}[t]{0,5\textwidth}
Brambory popatláme směsí z koření a vrazíme do trouby

Až budou prakticky hotové, rozpálíme pánev a na olivovém oleji osmahneme
\end{minipage}
\pagebreak
\section{Ryby}

\subsection{Rybí karbenátky s kari}
\begin{minipage}[t]{0,5\textwidth}
\begin{myitemize} 
\item 1kg ryby bez kostí, filety, filé
\item 4 vajíčka
\item 2 lžičky zelené kari pasty
\item zelenisko (petržel, pažitka)
\item trošku strouhanky
\item limetky
\item česnek nakrájený na kousky
\end{myitemize}
\end{minipage}
\begin{minipage}[t]{0,5\textwidth}
Rybu nakrájíme na malé kousíčky a kus rozmixujeme jako spojovací hmotu. Smícháme s kari pastou vajíčky, strouhankou, zeleniskem, česnekem a limetkou. Dobře promícháme a ted se to dá osmažit, nebo pokud jste odvážní a máte dobrou směs tak i na grilu.
\end{minipage}
\pagebreak
\section{Zmrzlina}
\subsection{}
\begin{minipage}[t]{0,5\textwidth}
\begin{myitemize} 
\item 600ml šlehačky
\item 150ml mlíka
\item 2ks vanilkového cukru s 5\% vanilky
\item 80g cukru
\item 5 žloutků
\end{myitemize}
\end{minipage}
\begin{minipage}[t]{0,5\textwidth}
Suroviny dáme do hrnce a přivedeme k varu. Nechat vychladit a umístnit do zmrzlinovače.
\end{minipage}
\end{document}
%template
\iffalse

\subsection{}
\begin{minipage}[t]{0,5\textwidth}
\begin{myitemize} 

\item 
\end{myitemize}
\end{minipage}
\begin{minipage}[t]{0,5\textwidth}

\end{minipage}

\fi